\section{Discussion}
\label{sec:discussion}
The expert survey results provide compelling evidence for the high perceptual realism of brightfield microscopy images generated by our unconditional diffusion model.
The near-chance level accuracy achieved by microscopy experts in distinguishing synthetic images from real ones strongly supports the notion that diffusion models can synthesize microscopy data that is visually indistinguishable from real-world acquisitions.
This addresses our first sub-research question and highlights the potential of diffusion models to generate data suitable for augmenting or even substituting real microscopy images in certain applications.

Object detection experiments further reveal the practical utility of diffusion-based synthetic data for single-cell detection and, importantly, highlight the benefits of increasing dataset size.
The comparable, and in some cases, improved performance of models trained with synthetic data—both when replacing parts of the real data (\textbf{replacement datasets})
and when adding to it (\textbf{augmentation datasets})—demonstrates at mAP\@50 that synthetic images can effectively capture essential cell features.
This directly addresses our main research question, suggesting that synthetic data is a valuable asset in training robust cell detection models, particularly when labeled real data is limited.

Specifically, the results from the \textbf{augmentation datasets} underscore the advantage of expanding a dataset with synthetic data.
We observed that models trained on these augmented datasets consistently achieved performance at least comparable to, and often slightly better than, the baseline across various metrics, especially mAP\@50 and mAP\@75.
This improvement, while sometimes marginal, is notable considering it comes without additional real image acquisition.
The larger, more diverse training set likely allowed the models to learn more robust features and generalize better to unseen test data, as the added synthetic data effectively contributed to learning without introducing detrimental artifacts.

The subtle performance enhancements might be attributed to the increased variability from the synthetic data generation process, which improves model generalization to variations in real-world images.
Furthermore, the larger dataset size inherently provides more training examples, potentially leading to better parameter optimization.
This is consistent with the principle that larger, relevant datasets often lead to improved deep learning model performance—a point reinforced by the expert survey confirming the realism of our synthetic images.

However, a limitation emerges at higher IoU thresholds (mAP\@75, mAP\@50:95).
The subtle performance decrease observed with higher proportions of synthetic data, even in the \textbf{augmentation datasets}, indicates a potential deficit in the fidelity of synthetic images for capturing fine cell boundary details.
While adding synthetic data generally benefits performance by increasing dataset size, its inherent limitations in replicating all nuances of real microscopy data remain a factor for tasks requiring high localization precision.
The observed sensitivity of RT-DETR models to synthetic data proportions also warrants further investigation, potentially pointing to architectural differences in how transformer-based models leverage synthetic data compared to CNN-based models.

The implications for biological research are significant.
The ability to generate high-quality synthetic images and use them to train cell detection models opens new avenues for addressing data scarcity and annotation bottlenecks.
This offers a cost-effective and scalable approach to augment limited datasets, potentially democratizing access to advanced cell detection techniques.
Synthetic data can reduce dependence on time-consuming manual labor and, by enabling robust label-free detection, aligns with best practices for minimizing phototoxicity in live-cell imaging.
Moreover, it allows for the creation of datasets with rare cell phenotypes or challenging conditions that are difficult to capture experimentally.
For these benefits to be realized responsibly, transparency in reporting the use of synthetic data is crucial for maintaining scientific integrity.
